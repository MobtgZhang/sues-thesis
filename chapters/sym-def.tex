%% 符号和缩略词说明
\chapter*{符号和缩略词说明}

这里是符号和缩略词段落,对文中所用符号缩略词所表示的意义及单位(或量纲)的说明。在目录中不出现。

中文采用宋体,英文及数字采用Times New Roman字体,小四,1.5倍行间距。若不需要说明,则删除此页。

一般这里采用的是三线表的方式表述一些符号和缩略词,在学术文章中经常使用到的一种表格。
插入表格需要用到table,tabularx包用于插入表格,一般最上面线和最下面线宽度在1.5pt左右,中间的线为0.75左右。
表格线宽度设置需要用到booktabs包引用三条线toprule,midrule和bottomrule三类线。
下面表\ref{tab:three_lines_table}是一个三线表简单的例子

\begin{table}[htp!]
    \centering
    \caption{一个三线表的示例}
    \label{tab:three_lines_table}
    \renewcommand\arraystretch{1.5} %定义表格高度
    \begin{tabularx}{0.9\textwidth}{p{2cm}<{\centering}p{6cm}<{\raggedright}}
        \toprule[1.5pt]
            符号    &   \quad 意义 \\
            \midrule[0.75pt]
            $ a $  &  符号1的意义    \\
            \hline
            $ b $  &  符号2的意义    \\
            \hline
            $ c $  &  符号3的意义符号3的意义    \\
            \hline
            $ d $  &  符号4的意义    \\
            \hline
            $ e $  &  符号5的意义     \\
        \bottomrule[1.5pt]
    \end{tabularx}
\end{table}

