\documentclass{suesreport}
\facultyname{这里是填写学院名称}
\subjectname{这里是填写学科名称}
\classnumber{这里是填写学号名称}
\studentname{这里是学生姓名}
\supervisor{这里是导师姓名}
\datetime{\today}
\contenttitle{这是我的中文题目}
\begin{document}
    \pagestyle{empty}
    % 封面
    \makecover
    % 摘要
    \begin{abstract}
        (摘要内容)摘要部分要求是:宋体,五号,1.25倍行间距,首行缩进2字符,500字左右。

        关键词部分,宋体,五号,关键词数量4-6个,用逗号分隔,末尾不打标点符号

        \keywords{关键词A,关键词B,关键词C}
    \end{abstract}
    % 正文内容
    \begin{center}
        \heiti\sanhao{正\qquad 文}

        \wuhao\qquad
    \end{center}
    \section{选题的来源、背景和意义}
    \subsection{这是二级标题}
    \subsubsection{这是三级标题}
     (内容)

     \begin{enumerate}[label=\arabic*. ]
        \item 图:图题字体为五号楷体。引用图应在图题右上角标出文献来源。
        
        图号按顺序全文通排,如图1,图2等。如果图中含有几个不同部分,应将分图号标注在分图的左上角,并在图题下列出各部分内容。
        
        绘图必须工整、清晰、规范。其中机械零件图按机械制图规格要求;示意图应能清楚反映图示内容;照片应在右下角给出放大标尺;实验结果曲线图应制成方框图。
        \item 表格:表格按顺序全文通排,如表1,表2等。表应有标题,字体为五号楷体,表内必须按规定的符号标注单位。
        \item 公式:公式书写应在文中另起一行。公式后应注明序号,该序号按顺序全文通排。
     \end{enumerate}
    \section{文献综述}

    \section{研究内容和创新点}
    \section{研究方法和设计方案}
    \section{研究重点、难点及解决方案}
    \section{完成学位论文的计划安排}
    \section{经费落实情况}
    \section{预期学术成果及应用价值}
    \section{参考文献}

    参考文献格式:
    \begin{enumerate}[label=\arabic*. ]
        \item 按论文中参考文献出现的先后顺序用阿拉伯数字连续编号,将序号置于方括号内,并视具体情况将序号作为上角标,或作为论文的组成部分。如:“……李××[1]对此作了研究,数学模型见文献[2]。”
        \item 参考文献中每条项目应齐全。文献中的作者不超过三位时全部列出;超过三位时一般只列前三位,后面加“等”字或“et al”;作者姓名之间用逗号分开;中外人名一律采用姓在前,名在后的著录法。
    \end{enumerate}

    
    参考文献中著录格式示例:
    \begin{enumerate}[label=(\arabic*) ]
        \item 期刊
        
        [序号] 作者.题名[J].刊名,出版年份,卷号(期号):起止页码.
        \item 专著
        
        [序号] 作者.书名:版本(第1版不标注) [M].出版地:出版者,出版年:起止页码.

        \item 论文集
        
        [序号] 主编.论文集名[C].出版地:出版者,出版年:起止页码.

        \item 学位论文
        
        [序号] 作者.题名[D].出版地:出版者,出版年:起止页码.

        \item 专利
        
        [序号] 专利申请者.题名:专利号[P].公告日期.

        \item 技术标准
        
        [序号] 起草责任者.标准名称:标准代号 标准顺序号-发布年[S].出版地:出版者,出版年度: 起止页码.
    \end{enumerate}

    举个例子,这里文献\cite{2004PSO_ZhangLibiao}中说明粒子群算法对多目标优化问题具有较好的性质。注意编译的时候需要使用xetex->bibtex->xetex->xetex的编译方式进行编译。
    \section{贡献}
    本模板是本人按照学校提供的word模板修改而成,希望大家能在学习latex中进步,支持一下作者!

    %% 参考文献使用
    %设置参考文献风格,参照使用 https://github.com/Haixing-Hu/GBT7714-2005-BibTeX-Style
    \bibliographystyle{gbt7714-2005}
    \nocite{*} 
    \bibliography{refers}
\end{document}

