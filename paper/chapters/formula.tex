% 定义一些常用的颜色标注
\newcommand{\red}[1]{{\textcolor{red}{#1}}}%
% 定义一个数学公式
\newcommand{\argmin}{\mathop{\mathrm{argmin}}\limits}
\newcommand{\argmax}{\mathop{\mathrm{argmax}}\limits}
\chapter{公式、图表}
\section{公式}

方便快捷写入公式是\LaTeX 相对于Word编辑器最为主要的优势之一,特别是熟练掌握之后,在输入公式的时候具有非常大的提升效果。


\LaTeX 中的公式分为两类,包括有\red{行内公式}和\red{行间公式},例如这是一个行间公式$f(x)=\frac{1}{\sqrt{2\pi}\sigma}\exp\left(-\frac{(x-\mu)^{2}}{2\sigma^{2}}\right)$,下面举例几个行间公式

\begin{eqnarray}
    f(x)&=&\dfrac{1}{\sqrt{2\pi}\sigma}\exp\left(-\dfrac{(x-\mu)^{2}}{2\sigma^{2}}\right)    
\end{eqnarray}

例如,定义一个分段函数
\begin{eqnarray}
    f(x)&=&\begin{cases}
        -x^{3}+x+8&,x\leq{2}\\
        \dfrac{1}{2}x^{2}&,2<x\leq{10}\\
        x+10&,x>10
    \end{cases}
\end{eqnarray}

也可以定义一个多行的连等的等式,定义如下所示
\begin{eqnarray}
    \cos{2x}&=\cos^{2}x-\sin^{2}x\\
        &=2\cos^{2}x-1\\
        &=1-2\sin^{2}
\end{eqnarray}

可以将多个等式对齐写在同一个语句块当中,例如麦克斯韦方程组

积分形式:
\begin{equation}
    \begin{cases}
        \displaystyle\oint_{l}\mathbf{H}\cdot{d}\mathbf{l}&=\displaystyle\iint_\mathbf{S}J\cdot{d}\mathbf{S}+\displaystyle\iint_{S}\dfrac{\partial\mathbf{D}}{\partial{t}}\cdot{dS}\\
        \displaystyle\oint_{l}\mathbf{E}\cdot{d}\mathbf{l}&=-\displaystyle\iint_{S}\dfrac{\partial\mathbf{B}}{\partial{t}}\cdot{d}\mathbf{S}\\
        \displaystyle\oint_{S}\mathbf{B}\cdot{d}\mathbf{S}&=0\\
        \displaystyle\oint_{S}\mathbf{D}\cdot{d}\mathbf{S}&=\displaystyle\iiint_\mathbf{V}\rho{d}\mathbb{V}
    \end{cases}
\end{equation}

微分形式:
\begin{equation}
    \begin{cases}
        \nabla\times\mathbf{H}&=J+\dfrac{\partial\mathbf{D}}{\partial{t}}\\
        \nabla\times\mathbf{E}&=-\dfrac{\partial\mathbf{B}}{\partial{t}}\\
        \nabla\cdot\mathbf{B}&=0\\
        \nabla\cdot\mathbf{H}&=\rho\\
    \end{cases}
    \label{equ:diff-function}
\end{equation}

带有矩阵定义的公式:
\begin{equation}
    \mathbf{H} = -\mathbf\mu \cdot \mathbf{B} = -\gamma B_o \mathbf{S}_z = -\frac{\gamma B_o\hbar}{2} 
        \begin{bmatrix}
            1& \cdots &1\\ 
            \vdots & \ddots & \vdots \\
            1 & \cdots & 1 
        \end{bmatrix}.
    \label{equ:matrix}
\end{equation}

在求解凸优化问题的时候,问题研究最后求解归结为以下的方程形式:
\begin{equation}
    \argmin_{x_{j},j=1,\cdots,N}\sum\limits_{j=1}^{N}c_{j}x_{j}
\end{equation}
\begin{eqnarray}
    \text{s.t.}\begin{cases}
        \sum\limits_{j=1}^{N}a_{ij}x_{j}=b_{i},&i=1,\cdots,{m}\\
        x_{j}\geq{0},
    \end{cases}
\end{eqnarray}

在文章当中每一个公式的后面均可以添加一个label的标签,这样就可以应用公式了,例如\cref{equ:matrix}就是刚刚我们表达的矩阵表达式。


