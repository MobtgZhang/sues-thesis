\chapter{算法、数学定义等}
\section{算法}

有时候我们在论文中提到一种算法需要插入代码和一些流程代码来详细说明我们所提出的算法内容,或者是说需要插入一个伪代码片段来说明算法的流程,就此下面详细说明这一个问题。
\subsection{插入代码}
\label{subsec:algorithms}
%<*tag>
这里我们举例四种语言,即C/C++语言,Java语言,Python语言和MATLAB语言。
%</tag>
\begin{itemize}
    \item C 语言
\begin{CLanguage}
# include<sdtio.h>
int main(){
    printf("This is C language block.");
    return 0;
}
\end{CLanguage}
\item C++ 语言
\begin{CPlusPlus}
# include<iostream>
int main(){
    std:::cout<<"This is C++ programing language."<<std::endl;
    return 0;
}
\end{CPlusPlus}
    \item Java 语言
\begin{Java}
public class Main{
    public static void main(String [] args ){
        System.out.printf("This is a Java programing.");
    }
}
\end{Java}
    \item Python 语言
\begin{Python}
def partition(arr,low,high): 
    i = ( low-1 )         # 最小元素索引
    pivot = arr[high]     
    for j in range(low , high): 
        # 当前元素小于或等于 pivot 
        if   arr[j] <= pivot: 
            i = i+1 
            arr[i],arr[j] = arr[j],arr[i] 
    arr[i+1],arr[high] = arr[high],arr[i+1] 
    return ( i+1 ) 
# arr[] --> 排序数组
# low  --> 起始索引
# high  --> 结束索引
# 快速排序函数
def quickSort(arr,low,high): 
    if low < high: 
        pi = partition(arr,low,high) 
        quickSort(arr, low, pi-1) 
        quickSort(arr, pi+1, high) 
arr = [10, 7, 8, 9, 1, 5] 
n = len(arr) 
quickSort(arr,0,n-1) 
print ("排序后的数组:") 
for i in range(n): 
    print ("%d" %arr[i]),
\end{Python}  
\item Matlab 语言
\begin{Matlab} 
    kk=2;[mdd,ndd]=size(dd);
    while ~isempty(V)
        [tmpd,j]=min(W(i,V));tmpj=V(j);
        for k=2:ndd
            [tmp1,jj]=min(dd(1,k)+W(dd(2,k),V));
            tmp2=V(jj);tt(k-1,:)=[tmp1,tmp2,jj];
        end
        tmp=[tmpd,tmpj,j;tt];[tmp3,tmp4]=min(tmp(:,1));
        if tmp3==tmpd
            ss(1:2,kk)=[i;tmp(tmp4,2)];
        else
            tmp5=find(ss(:,tmp4)~=0);tmp6=length(tmp5);
            if dd(2,tmp4)==ss(tmp6,tmp4)
                ss(1:tmp6+1,kk)=[ss(tmp5,tmp4);tmp(tmp4,2)];
            else
                ss(1:3,kk)=[i;dd(2,tmp4);tmp(tmp4,2)];
            end
        end
        dd=[dd,[tmp3;tmp(tmp4,2)]];V(tmp(tmp4,3))=[];
        [mdd,ndd]=size(dd);kk=kk+1;
    end
    S=ss;D=dd(1,:);
    \end{Matlab}
\end{itemize}

\subsection{插入伪代码}

下面是一个粒子群算法的伪代码
\begin{center}
    \begin{minipage}{0.8\textwidth}
        \begin{algorithm}[H]%[!htp]
            \caption{粒子群算法} %算法的名字
            {\bf 输入:} %算法的输入, \hspace*{0.02in}用来控制位置,同时利用 \\ 进行换行
            群体规模$N$,每个粒子的位置$x_{i}$和速度$v_{i}$\\
            {\bf 输出:} %算法的结果输出
            output result
        \begin{algorithmic}[1]
            \State 初始化粒子群 % \State 后写一般语句
            \While{不满足结束条件} % For 语句,需要和EndFor对应
                \State 计算每个粒子的适应度$F_{it}(i)$
                \State 对每个粒子,用它的适应度值$F_{it}(i)$和个体极值$P_{\text{best}}(i)$比较,如果$F_{it}(i)>P_{\text{best}}(i)$,则用$F_{it}(i)$替换掉$P_{\text{best}}(i)$;
                \State 对每个粒子,用它的适应度值$F_{it}(i)$和全局极值$g_{\text{best}}$比较,如果$F_{it}(i)>g_{\text{best}}$,则使用$F_{it}(i)$替换$g_{\text{best}}$;
                \State 更新例子的速度$v_{i}$和位置$x_{i}$。
            \EndWhile
            \State \Return result
            \end{algorithmic}
        \end{algorithm}
    \end{minipage}
    \label{algo:pso_algoritm}
\end{center}
\section{数学定义等}

在文章当中,有些地方需要一些数学的定义、解释、证明等等,这需要一定的格式,这里提供了以下若干个数学当中需要的命令:

下面简单介绍一下定理、证明等环境的使用
\begin{definition}[$\varepsilon-\delta$极限定义]
	如果对于$\forall\varepsilon>0$(不论它多么小),$\exists\delta>0$,是的对于适合不等式
    \begin{equation}
        0<\left|x-x_{0}\right|<\delta
    \end{equation}
    的一切$x$,对应的函数值$f(x)$均满足不等式
    \begin{equation}
        \left|f(x)-A\right|<\varepsilon
    \end{equation}
    那么常数$A$就叫做函数$f(x)$当$x\rightarrow{x_{0}}$时的极限,记作
    \begin{equation}
        \lim\limits_{x\rightarrow{x_{0}}}f(x)=A\text{  Or } f(x)\rightarrow{A}\text{ When }x\rightarrow{x_{0}}
    \end{equation}
	\label{def:nosense}
\end{definition}
\cref{def:nosense}从根本上定义了极限。

除了 definition 环境,还可以使用 theorem 、lemma、corollary、assumption、conjecture、axiom、principle、problem、example、proof、solution 这些环境,根据论文的实际需求合理使用。

我们通过以下的例子举例说明。

\begin{theorem}[勾股定理]
	在平面上的一个直角三角形中,两个直角边边长(a,b)的平方加起来等于斜边长(c)的平方。
	\label{thm:example}
\end{theorem}
由\cref{thm:example}我们知道了勾股定理的使用。

接下来看下面这两个定义

\begin{definition}[二次剩余]
    当存在某个$X$,表达式$X^{2}\equiv{d}(\mod{p})$ 成立的时候,称$d$是模$p$的二次剩余;当对任意$X,X^{2}\equiv{d}(\mod{p})$不成立的时候,称$d$是模$p$的非二次剩余。
    \label{def:QuadraticResidue}
\end{definition}

\begin{definition}[Legebdre符号]
    设奇素数$p$,定义整变数$d$的函数
    \begin{equation}
        \left(\dfrac{d}{p}\right)=\begin{cases}
            1&,\text{d是p的二次剩余}\\
            -1&,\text{d是p的非二次剩余}\\
            0&,p|d
        \end{cases}
    \end{equation}
    \label{def:Legebdre}
\end{definition}

另外,关于Legebdre符号有以下的定理
\begin{theorem}[Legebdre符号]
    Legebdre符号有以下的性质
    \begin{enumerate}[label=\arabic*.\quad]
        \item $\left(\dfrac{d}{p}\right)=\left(\dfrac{d+p}{p}\right)$;
        \item $\left(\dfrac{d}{p}\right)\equiv{d^{(p-1)/2}}(\mod{p})$;
        \item $\left(\dfrac{dc}{p}\right)=\left(\dfrac{d}{p}\right)\left(\dfrac{c}{p}\right)$;
        \item 若$p\nmid{d}$,则$\left(\dfrac{d^{2}}{p}\right)=1$;
        \item $\left(\dfrac{1}{p}\right)=1$;
        \item $\left(\dfrac{-1}{p}\right)=(-1)^{(p-1)/2}$。
    \end{enumerate}
    \label{thm:Legebdre}
\end{theorem}

另外一个重要的性质,通过\cref{thm:LegebdreExtension}所示
\begin{theorem}
    若$p,q$均为奇素数,$p\neq{q}$,那么就会有$\left(\dfrac{q}{p}\right)\cdot\left(\dfrac{p}{q}\right)=(-1)^{(p-1)/2\cdot{(q-1)/2}}$。
    \label{thm:LegebdreExtension}
\end{theorem}

基于以上的这些个性质,研究一些定理

\begin{lemma}
	设素数$p>3$,那么不定方程$x^{2}+3y^{2}=p$有解的充要条件是$\left(\dfrac{-3}{p}\right)$,即$p$是形如$6k+1$的素数。
	\label{lem:example}
\end{lemma}
由\cref{lem:example}我们知道了引理。所以证明如下所示
\begin{proof}[证明\cref{lem:example}]
	必要性证明:若上述不定方程有解$x_{0},y_{0}$,则显然有
    \begin{equation}
        \left(x_{0},3y_{0}\right)=\left(p,3x_{0}y_{0}\right)=1
    \end{equation}
    使用上述定理可以得到
    \begin{equation}
        1=\left(\dfrac{x_{0}^{2}}{p}\right)=\dfrac{-3y_{0}^{2}}{p}=\left(\dfrac{-3}{p}\right)
    \end{equation}
    根据\cref{thm:Legebdre}和\cref{thm:LegebdreExtension}可以得到
    \begin{equation}
        \left(\dfrac{-3}{p}\right)=\left(\dfrac{-1}{p}\right)\left(\dfrac{3}{p}\right)=(-1)^{(p-1)/2}\cdot(-1)^{(p-1)/2}\left(\dfrac{p}{3}\right)=\left(\dfrac{p}{3}\right)
    \end{equation}
    所以这样就会得到$\left(\dfrac{p}{3}\right)$,即$p$满足$p=6k+1$。

    充分性证明有些许复杂


    由$\left(\dfrac{-3}{p}\right)$可以得到同余方程$s^{2}\equiv{-3}(\mod{p})$必定有解。
    设$s_{0}$为其解,考虑集合$\left\{(u,v)\left|0\leq{u}<\sqrt{p},0\leq{v}<\sqrt{p}\right.\right\}$,
    这个集合的元素个数等于$\left(\left[\sqrt{p}\right]+1\right)^{2}>p$.由抽屉原则可以得到,必定有两组不同的$\left\{u_{1},v_{1}\right\},\left\{u_{2},v_{2}\right\}$,使得
    \begin{equation}
        s_{0}v_{1}-u_{1}\equiv{s_{0}v_{2}-u_{2}}(\mod{p})
    \end{equation}
    即有$\hat{u}\equiv{s_{0}\hat{v}}(\mod{p})$这里$\hat{u_{1}}-\hat{u_{2}},\hat{v_{1}}-\hat{v_{2}}$一定不全为零。由此以及$s_{0}$是上述同余方程的解,即得到
    \begin{equation}
        \hat{u}^{2}+3\hat{v}\equiv{0}(\mod{p})
    \end{equation}
    另一方面,由原方程可以得到
    \begin{equation}
        0<\hat{u}^{2}+3\hat{v}<4p
    \end{equation}
    所以由上述两式可以得到有以下几种可能:
    \begin{equation}
        \hat{u}^{2}+3\hat{v}=p,2p,3p
    \end{equation}
    若$\hat{u}^{2}+3\hat{v}=p$,则原方程必有解;若$\hat{u}^{2}+3\hat{v}=2p$,则必有$2\mid{\hat{u}+\hat{v}}$,所以就会有$2\mid{\hat{u}^{2}+3\hat{v}^{2}}$,因而得到$2\mid{p}$,与$p$是奇质数矛盾;若$\hat{u}^{2}+3\hat{v}=3p$,则$3\mid{3\hat{u}}$,因而就会有
    \begin{equation}
        \hat{v}^{2}+3\left(\dfrac{\hat{u}}{3}\right)=p
    \end{equation}
    这样原方程也有解.证明完毕.
	\label{prf:example}
\end{proof}
由\cref{prf:example}我们知道了证明环境的使用。通过\cref{lem:example},我们可以得到以下的推论

\begin{corollary}
	设$2\nmid{s}$,那么$s^{3}=a^{2}+3b^{2},(a,b)=1$成立的虫咬条件是$\exists\alpha,\beta$,使得
    \begin{equation}
        \begin{cases}
            s&=\alpha^{2}+3\beta^{2}\\
            a&=\alpha^{3}-9\alpha\beta^{2}\\
            b&=3\alpha^{2}-3\beta^{3}
        \end{cases}
    \end{equation}
	\label{cor:example}
\end{corollary}
由\cref{cor:example}我们知道了推论环境的使用。
 
\begin{assumption}[假设$m$和$n$不存在]
	这是一个假设。
	\label{asu:example}
\end{assumption}
由\cref{asu:example}我们知道了假设环境的使用。
 
\begin{conjecture}[霍奇猜想]
	给定一个随机形状,它可以和一个多项式描述的形状同胚。换句话说,我们能够取任意曲面并通过适当变形操作,最终得到一个多项式的解集
	\label{con:example}
\end{conjecture}
由\cref{con:example}我们知道了猜想环境的使用。
 
\begin{axiom}[平行公理]
	经过直线外一点,有且只有一条直线与这条直线平行
	\label{axi:example}
\end{axiom}
由\cref{axi:example}我们知道了公理环境的使用。


\begin{corollary}[直线平行推论]
    如果两条直线都与第三条直线平行,那么这两直线互相平行。
\end{corollary}

\begin{principle}[能量守恒定律]
	能量既不会凭空产生,也不会凭空消失,它只会从一种形式转化为另一种形式,或者从一个物体转移到其它物体,而能量的总量保持不变。
	\label{pri:example}
\end{principle}
由\cref{pri:example}我们知道了定律环境的使用。
 
\begin{problem}[理发师悖论]
	假如理发师,声称他只给给所有自己不刮胡子的人刮胡子。那么我们可以问: 他是否应该给自己刮胡子 ?
	\label{pro:example}
\end{problem}
由\cref{pro:example}我们知道了问题环境的使用。
 
\begin{example}[复积分计算]
	计算$\displaystyle\oint_{C}\dfrac{dz}{\left(z-z_{0}\right)^{2}}$,其中$n$为任意整数,$C$为以$z_{0}$为中心,$r$为半径的圆周。
	\label{exa:example}
\end{example}
由\cref{exa:example}我们知道了例子环境的使用。
\begin{solution}[例子\cref{exa:example}]
	C的参数方程为$z=z_{0}+re^{i\theta}$,$0\leq\theta{2\pi}$,$dz=ire^{i\theta}d\theta$。

    故而有
    \begin{eqnarray}
        \oint_{C}\dfrac{dz}{\left(z-z_{0}\right)^{n}}&=&\displaystyle\int_{0}^{2\pi}\dfrac{ire^{i\theta}}{r^{n}e^{in\theta}}d\theta\nonumber\\
        &=&\dfrac{i}{r^{n-1}}\displaystyle\int_{0}^{2\pi}e^{-i(n-1)\theta}d\theta\nonumber\\
        &=&\dfrac{i}{r^{n-1}}\displaystyle\int_{0}^{2\pi}\left[\cos{(n-1)\theta}+i\sin{-(n-1)\theta}\right]d\theta\nonumber\\
        &=&\dfrac{i}{r^{n-1}}\displaystyle\int_{0}^{2\pi}\cos{(n-1)\theta}d\theta+\dfrac{1}{r^{n-1}}\displaystyle\int_{0}^{2\pi}\sin{(n-1)\theta}d\theta\nonumber\\
        &=&\begin{cases}
            2\pi{i},&(n=1)\\
            0,&(n\neq{1})
        \end{cases}\nonumber
    \end{eqnarray}
	\label{sol:example}
\end{solution}

由\cref{sol:example}我们知道了解环境的使用。

