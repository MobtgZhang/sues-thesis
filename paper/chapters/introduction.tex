\chapter{绪论}
\section{关于\LaTeX 模板}
本模板是本人为写硕士学位论文而写的,本项目地址代码均在\href{https://github.com/mobtgzhang/sues-thesis}{GitHub}上。论文模板写得稍微有一些简陋,但是足够用于完成上海工程技术大学的硕士学位论文。

\section{关于\TeX 和 \LaTeX}

{\TeX}是由图灵奖得主,程序(program)和算法(algorithm)这两个概念的
提出者,《计算机程序设计的艺术》(The Art of Computer Programming)的作者,著名计算机科学家
Donald E. Knuth(高德纳)发明的排版系统。TeX是特别优秀的排版工具,尤其善于处理复杂的图表和公式。
%
\href{https://en.wikipedia.org/wiki/LaTeX}{\LaTeX}(拉泰赫)是一种基于{\TeX}的排版系统,由由美国计算机学家Leslie Lamport(莱斯利·兰伯特)在20世纪80年代初期开发,
因此被称为Lamport Tex,简称LaTeX。

\section{关于使用的平台和发行版本}
\LaTeX 拥有众多的发行版,主要有一下几个:
\begin{table}[!h]
  \centering
  %\renewcommand{\arraystretch}{1}
  \setlength\tabcolsep{6.4pt}
  \bicaption{主要的\LaTeX{} 发行版。}{Main \LaTeX{} distributions.}\label{tab:latex-distr}
  \begin{tabular}{c|c|c|c}
    \hline
    \diagbox{发行版}{支持平台} & Windows & Linux & OSX \\
    \hline
    \href{http://www.tug.org/texlive/}{TexLive} & \cmark  & \cmark &  \cmark \\
    \hline
    \href{https://miktex.org/}{MikTex} & \cmark  & \xmark & \xmark  \\
    \hline
    \href{http://www.tug.org/mactex/}{MacTex} & \xmark  & \xmark & \cmark  \\ \hline
    \end{tabular}\vspace{-6pt}
  
\end{table}%

我比较推荐TexLive,因为它支持主流的平台,而且更新频率也比较高。
\section{使用哪个\TeX 编辑器}
目前\TeX 编辑器是多种多样的,使用一个较好的编辑器对于编写\LaTeX 文档是有着非常好的效果。TexLive默认的编辑器是texstudio编辑器,
但是笔者更喜欢VSCode作为编辑器,它具有非常好的联想功能和代码主题,看起来也非常省眼。

\paragraph{在线编辑器} 现在有很多在线的latex编辑器,最为常见和著名的编辑器要属\href{https://overleaf.com/}{OverLeaf}。
这种在线latex平台编辑器能够非常好的支持各种排版的功能,类似于腾讯文档,它支持多人协作编辑文档,具有较好的使用效果。




