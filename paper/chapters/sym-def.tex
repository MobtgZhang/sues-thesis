\begin{center}
    \heiti\sanhao\bfseries{符号和缩略词说明}
    \vspace{1cm}
\end{center}

对文中所用符号缩略词所表示的意义及单位(或量纲)的说明。在目录中不出现。若不需要说明,则删除此页。

这里使用三线表举个例子。这里的符号描述可以使用三线表描述一些符号和缩略词,在学术文章当中经常使用到的一种表格。插入表格,一般最上面线和最下面线宽度为1.5pt,中间线条宽度为0.75pt。表格线宽度设置需要用到booktabs包,这个包当中包含有toprule、midrule和bottomrule三类线。下面\cref{tab:latex-sys-def}是一个三线表简单的例子
\begin{table}[hbpt]
    \centering
    \linespread{1.2}
    \caption{一个三线表的示例}
    \begin{tabular}{p{0.2\textwidth}<{\centering}p{0.6\textwidth}<{\centering}}
        \toprule[1.5pt]
        符号表示 & 符号意义\\
        \midrule[0.75pt]
        $e$& 数学自然对数\\
        $\pi$& 数学圆周率\\
        $\epsilon$ & 介电值常数\\
        $G$ & 万有引力常数\\
        $k$ & 玻尔兹曼常数\\
        \bottomrule[1.5pt]
    \end{tabular}
    \label{tab:latex-sys-def}
\end{table}
\newpage
%
