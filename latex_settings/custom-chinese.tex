% 设置页边距
\usepackage{geometry}
\geometry{a4paper,top=3cm,bottom=2.5cm,left=3cm,right=2.5cm}

% 设置行间距 1.5倍
\linespread{1.25}\selectfont
% 设置段与段之间的垂直距离 \parskip默认橡皮长度是0pt plus 1pt
\setlength{\parskip}{0pt}
% \setlength{\parindent}{0pt}

%设置字体
% 设置英文字体
\setmainfont{Times New Roman}[
    BoldFont = Times New Roman Bold,
    ItalicFont = Times New Roman Italic,
    BoldItalicFont = Times New Roman Bold Italic
]

\setsansfont{Times New Roman}[
    BoldFont = Times New Roman Bold,
    ItalicFont = Times New Roman Italic,
    BoldItalicFont = Times New Roman Bold Italic
]

\setmonofont{Times New Roman}[
    BoldFont = Times New Roman Bold,
    ItalicFont = Times New Roman Italic,
    BoldItalicFont = Times New Roman Bold Italic
]

%设置字号
\usepackage{ctexsize,type1cm}
\newcommand{\yihao}{\fontsize{26pt}{39pt}\selectfont}
\newcommand{\xiaoyi}{\fontsize{24pt}{36pt}\selectfont}   
\newcommand{\erhao}{\fontsize{22pt}{33pt}\selectfont}          
\newcommand{\xiaoer}{\fontsize{18pt}{27pt}\selectfont}          
\newcommand{\sanhao}{\fontsize{16pt}{24pt}\selectfont}        
\newcommand{\xiaosan}{\fontsize{15pt}{22.5pt}\selectfont}        
\newcommand{\sihao}{\fontsize{14pt}{21pt}\selectfont}            
\newcommand{\xiaosi}{\fontsize{12pt}{18pt}\selectfont}            
\newcommand{\wuhao}{\fontsize{10.5pt}{15.75pt}\selectfont}
\newcommand{\xiaowu}{\fontsize{9pt}{13.5pt}\selectfont}    
\newcommand{\liuhao}{\fontsize{7.5pt}{11.25pt}\selectfont}

%使用公式,表格,图片
\usepackage{mathtools,amsmath,amssymb,graphicx,array,float}

% 设置页眉面脚
%% 设置章节前的页码格式
\usepackage[pagestyles]{titlesec}
\newpagestyle{MyStyle}{
  \sethead{}{}{}
  \setfoot{}{\Roman{page}}{}
}

\usepackage{fancyhdr}
%重新设置headings
\fancypagestyle{headings}{
    \fancyhf{}
    \fancyfoot[C]{\songti\xiaowu 第 \thepage 页}
    \fancyhead[L]{\songti\xiaowu 上海工程技术大学硕士学位论文}
    \fancyhead[R]{\songti\xiaowu \leftmark }
}

%重新设置plain
\fancypagestyle{plain}{
    \fancyhf{}
    \fancyfoot[C]{\songti\xiaowu 第 \thepage 页}
    \fancyhead[L]{\songti\xiaowu 上海工程技术大学硕士学位论文}
    \fancyhead[R]{\songti\xiaowu \leftmark }
}

%设置双线页眉
\makeatletter
\def\headrule{
    {\if@fancyplain\let\headrulewidth\plainheadrulewidth\fi%
    \hrule\@height 2.0pt \@width\headwidth\vskip1pt%上面线为1pt粗
    \hrule\@height 0.5pt \@width\headwidth  %下面0.5pt粗
    \vskip-2\headrulewidth\vskip-1.2pt}    %两条线的距离1pt
    \vspace{6mm}}     %双线与下面正文之间的垂直间距
\makeatother

%设置双线页脚
\makeatletter
\def\footrule{
    {\if@fancyplain\let\footrulewidth\plainfootrulewidth\fi%
    \hrule\@height 0.5pt \@width\headwidth          %上面0.5pt粗
    \vskip 1pt
    \hrule\@height 2.0pt \@width\headwidth %下面线为1pt粗
    \vskip-2\headrulewidth\vskip-1.2pt}    %两条线的距离1pt
    \vspace{8mm}}     %双线与下面正文之间的垂直间距
\makeatother

%设置文章格式
\ctexset {
    contentsname={目\quad 录},
    listfigurename={插图},
    listtablename={表格},
    figurename={图},
    tablename={表},
    bibname={参考文献},
    appendixname={附\quad 录},
    chapter={
        beforeskip={0pt},
        nameformat={\heiti\sanhao\centering},
        titleformat={\heiti\sanhao\centering},
    },
    section={
        format={\heiti\sihao},
    },
    subsection={
        format={\heiti\xiaosi},
    },
    subsubsection={
        format={\heiti\xiaosi},
    }
}

% 目录中的章加点
\usepackage[titles]{tocloft}
\renewcommand{\cftdot}{$\cdot$}
\renewcommand{\cftdotsep}{1.5}
\setlength{\cftbeforechapskip}{10pt}

\renewcommand{\cftchapleader}{\cftdotfill{\cftchapdotsep}}
\renewcommand{\cftchapdotsep}{\cftdotsep}
\makeatletter
\renewcommand{\numberline}[1]{%
\settowidth\@tempdimb{#1\hspace{0.5em}}%
\ifdim\@tempdima<\@tempdimb%
  \@tempdima=\@tempdimb%
\fi%
\hb@xt@\@tempdima{\@cftbsnum #1\@cftasnum\hfil}\@cftasnumb}
\makeatother

% 设置目录字体尺寸
\renewcommand{\cftchapfont}{\heiti\xiaosi}
\renewcommand{\cftsecfont}{\heiti\wuhao}
\renewcommand{\cftsubsecfont}{\heiti\wuhao}

%使用代码排版包
\usepackage{listings}
\usepackage{color}
\lstset{%
    frame=shadowbox,
    extendedchars=false,            % 不使用xelatex而使用CJK方式处理汉字
    language=python,
    basicstyle=\sffamily,           % 设置整体格式
    keywordstyle=\bfseries,         % 关键字格式
    commentstyle=\rmfamily\itshape, % 注释格式
    stringstyle=\ttfamily,          % 字符串格式
    columns=flexible,
    escapechar=',                   % 注释中显示汉字,eg //'一个整数'
    tabsize=4,
    numbers=left,
    numberstyle=\small,             % 行号字体设置
    stepnumber=1,                   % 行号距离设置,1代表每行加行号
    numbersep=8pt,                  % 行号和代码距离设置
    backgroundcolor=\color{white},
    showspaces=false,               % show spaces adding particular underscores
    showstringspaces=false,         % 使用下划线连接字符串
    showtabs=false,
    frame=single,                   % 给代码加边框
    captionpos=b,                   % sets the caption-position to bottom
    breaklines=true,                % 自动换行设置
    breakatwhitespace=false,        % sets if automatic breaks should only happen at whitespace
    escapeinside={\%*}{*)},         % if you want to add a comment within your code
    xleftmargin=2em,                % 设置左边距,宽度默认是与页芯等宽的
    xrightmargin=2em,               % 设置右边距,宽度默认是与页芯等宽的
    aboveskip=1em                   % 设置上边距
}

%设置自定义变量
\newcommand\degree{^\cire}

% 定义文献引用格式,\cite正常引用 \supercite右上角引用
\usepackage{cite}
\newcommand{\upcite}[1]{\textsuperscript{\textsuperscript{\cite{#1}}}}
\newcommand\supercite[2][]{%
\textsuperscript{\cite[#1]{#2}}
}

\usepackage{enumitem}
\setlist[description]{
    itemsep=-5pt,
    font=\songti,
}

% 定义中文封面环境
\newenvironment{titletabbing}
{\par\bfseries\songti\sihao\tabbing}
{\endtabbing\par}

\usepackage[nottoc]{tocbibind}
\endinput
